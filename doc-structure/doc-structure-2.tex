\documentclass[UTF8]{ctexart} % ctexbook, ctexrep

% 可以让生成的文章目录有链接,点击时会自动跳转到该章节。而且也会使得生成的pdf 文件带有目录书签。
\usepackage{hyperref} 
\hypersetup{hidelinks} % 没有该命令,目录上会有红色方框

\ctexset {
	section = {
		format+ = \zihao{-4} \heiti \raggedright,
		name = {第,章、},
		number = \chinese{section},
		beforeskip = 1.0ex plus 0.2ex minus .2ex,
		afterskip = 1.0ex plus 0.2ex minus .2ex,
		aftername = \hspace{0pt}
	},
	subsection = {
		format+ = \zihao{5} \heiti \raggedright,
		% name = {thesubsection},
		name = {第,节、},
		number = \arabic{subsection},
		beforeskip = 1.0ex plus 0.2ex minus .2ex,
		afterskip = 1.0ex plus 0.2ex minus .2ex,
		aftername = \hspace{0pt}
	}
}

\begin{document}
	
	\tableofcontents % 产生文档目录

	\newpage % 分页
	
	%\chapter{章节} % 产生带章节的大纲,只能在book中
	\section{导言}
	本书是一部简明扼要的哲学入门,是写给第一次接触哲学的读者看的。一般说来,人们上大学之后才开始学习哲学,所以我想,本书的大多数读者都应该在读大学,或者已经从大学毕业。不过年龄问题与哲学学科的本质并无关系,要是有喜欢抽象观念和理论论证的聪明高中生能读到这本书,并且也能从中享受到哲学思考的乐趣,我会非常高兴的。
	
	% 为了保证源文件的清晰,分段通常是用插入空行来实现的。
	在系统学习关于世界的知识之前,我们的分析能力就已经高度发达了。\par 大概十四岁左右,许多人就开始以自己的方式思考哲学问题了:什么东西真的存在?
	
	% 反斜杠只是换行,并没有产生新的段落
	本书直接介绍九个哲学问题,其中每一个都可以被独立地理解,而不必涉及思想史。\\本书不打算讨论过去伟大的哲学著作,也不讲这些著作的文化背景。
	
	\section{外部世界是否存在?}
	\subsection{他人的意识}
	\subsubsection{身—心问题} % 在book中subsubsection命令不起作用
	\subsection{第二节}
%	\chapter{人生的意义}
	\subsection{词语的意义}
	\subsubsection{自由意志}
	\subsection{对与错}
	
\end{document}